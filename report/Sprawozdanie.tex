\documentclass{mwrep}

% Polskie znaki
\usepackage{polski}
\usepackage[utf8]{inputenc}
\usepackage[T1]{fontenc}
\usepackage{lmodern}
\usepackage{indentfirst}

% Strona tytułowa
\usepackage{pgfplots}
\usepackage{siunitx}
\usepackage{paracol}

% Pływające obrazki
\usepackage{float}
\usepackage{svg}
\usepackage{graphicx}

% table of contents refs
\usepackage{hyperref}
\usepackage{cleveref}
\usepackage{booktabs}
\usepackage{listings}


\SendSettingsToPgf
\title{\bf Sprawozdanie z bloku drugiego \vskip 0.1cm}
\author{Konrad Winnicki \\ Jakub Sikora}
\date{\today}
\pgfplotsset{compat=1.15}	
\begin{document}

\makeatletter
\renewcommand{\maketitle}{\begin{titlepage}
		\begin{center}{
				\LARGE {\bf Politechnika Warszawska}}\\
            \vspace{0.4cm}
            \leftskip-0.9cm
            {\LARGE {\bf \mbox{Wydział Elektroniki i Technik Informacyjnych}}}\\
            \vspace{0.2cm}
            {\LARGE {\bf \mbox{Instytut Automatyki i Informatyki Stosowanej}}}\\
            
            \vspace{5cm}
            \leftskip-0.5cm
			{\bf \Huge \mbox{Sterowanie i symulacja robotów} \vskip 0.1cm}
		\end{center}
		\vspace{0.1cm}

		\begin{center}
			{\bf \LARGE \@title}
		\end{center}

		\vspace{10cm}
		\begin{paracol}{2}
			\addtocontents{toc}{\protect\setcounter{tocdepth}{1}}
			\subsection*{Zespół:}
			\bf{ \Large{ \noindent\@author \par}}
			\addtocontents{toc}{\protect\setcounter{tocdepth}{2}}

			\switchcolumn \addtocontents{toc}{\protect\setcounter{tocdepth}{1}}
			\subsection*{Prowadzący:}
			\bf{\Large{\noindent mgr. inż. Wojciech \\ Dudek}}
			\addtocontents{toc}{\protect\setcounter{tocdepth}{2}}

		\end{paracol}
		\vspace*{\stretch{6}}
		\begin{center}
			\bf{\large{Warszawa, \@date\vskip 0.1cm}}
		\end{center}
	\end{titlepage}
}
\makeatother
\maketitle

\tableofcontents

\chapter{Laboratorium 1}

\section{Znalezione błędy w funkcjonowaniu systemu robota}

\section{Opis algorytmu interpolacji liniowej}

\subsection{Interpolacja na podstawie zadawanych prędkości}

\subsection{Interpolacja na podstawie danych odometrii}


\chapter{Projekt 1}

\section{Struktura oprogramowania stworzonego do zbierania danych}
Diagram strukturalny (zawiera węzły/procesy składowe systemu oraz komunikację między
nimi)
\section{Opis działania węzła zbierającego dane}
Opis z diagramami
\section{Opis działania węzła sterującego robota}
Opis z diagramami
\section{Sposób analizy danych}

\section{Wykresy i wnioski}


\chapter{Laboratorium 2}

\section{Stworzone środowisko i jego mapa}

\section{Przykładowe ścieżki zaplanowane w środowiskach}

\section{Pliki uruchomieniowe symulacji}


\chapter{Projekt 2}

\section{Struktura sterownika robota}

\section{Opis działania węzła planującego}

\section{Pliki konfiguracyjne map kosztów oraz lokalnego planera}

\section{Wyjaśnienie zastosowanych parametrów}
Dlaczego taki parametr ustawiono i dlaczego taka wartość?
\section{Weryfikacja działania}
Zrzuty ekranu z zaplanowaną i wykonaną ścieżką (Typ
wizualizacji: Odometry ): http://wiki.ros.org/rviz/DisplayTypes/Odometry
\end{document}